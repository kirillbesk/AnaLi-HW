\documentclass[11pt]{article}

\usepackage{amsfonts}
\usepackage{graphicx}
\usepackage{amssymb}
\usepackage{tikz}
\usepackage{amsmath}

\newcommand*\circled[1]{\tikz[baseline=(char.base)]{
            \node[shape=circle,draw,inner sep=2pt] (char) {#1};}}
\DeclareRobustCommand{\brkbinom}{\genfrac[]{0pt}{}}

\begin{document}
	\noindent \begin{huge}Haitham Abdel Razaq Moh'd Almatani  407920\\
	
	\noindent Themistoklis Dimaridis 355835\\
	
	
	\noindent Kirill Beskorovainyi 	451420\end{huge}\\\vspace{0.05in}
	
		\noindent \begin{Large}Aufgabe 1:\end{Large}\\[2pt]
			\indent a)\\
				$$T_1= \left\{\brkbinom{a}{b} \in \mathbb{R}^2 \mid 2x_1=x_2 \right\} \hspace{30pt} T_2= \left\{\brkbinom{a}{b} \in \mathbb{R}^2 \mid x_1^3+x_2^3=1 \right\}$$
				Wir verwenden das Teilraum kriterium. $T_1$, $T_2$ sind Teilräume des $\mathbb{R}^2$, falls diese den Nullvektor enthalten und abgeschlossen bezüglich Addition und skalarer Multiplikation sind.\\
				\underline{Über $T_1$ gilt:}\\
				\indent $\bullet$ $\vec{0}=\brkbinom{0}{0} \in T_1$, da $2 \cdot 0 = 0$\\
				\indent $\bullet$ Für $\brkbinom{x_1}{x_2}$, $\brkbinom{y_1}{y_2} \in T_1$ ist $x_2=2x_1$ und $y_2=2y_1$\\ 
				Dann gilt: $\brkbinom{x_1}{x_2}+\brkbinom{y_1}{y_2}=\brkbinom{x_1+y_1}{x_2+y_2}$ mit\\
				$$x_2+y_2=2x_1+2y_1=2(x_1+y_1)$$
				Also ist $\brkbinom{x_1}{x_2}+\brkbinom{y_1}{y_2} \in T_1$ und damit ist $T_1$ abgeschlossen bezüglich der Addition.\\
				\indent $\bullet$ Sei $\brkbinom{x_1}{x_2} \in T_1$, $\alpha \in \mathbb{R}$. Dann gilt per Definition:\\
				$$x_2=2x_1$$
				Also ist für $\alpha \brkbinom{\alpha x_1}{\alpha x_2}$, dass $\alpha x_2=\alpha (2x_1)=2(\alpha x_1)$\\
				Damit ist $\alpha \brkbinom{x_1}{x_2} \in T_1$ und $T_1$ ist auch abgeschlossen bezüglich skalarer Multiplikation.\\
				Damit ist $T_1$ ein Teilraum des $\mathbb{R}^2$.\\
				\underline{Über $T_2$ gilt:}\\
				\indent $\bullet$ $\vec{0}=\brkbinom{0}{0} \notin T_1$, da $0^3+0^3=0 \neq 1$\\
				$=>$ Somit ist der Nullvektor nicht in $T_2$ enthalten und damit ist $T_2$ kein Teilraum des $\mathbb{R}^2$.\\
\newpage
			\indent b)\\
				$$V=\{f: \mathbb{R} \longrightarrow \mathbb{R}\}$$
				$$T=\{f: \mathbb{R} \longrightarrow \mathbb{R} \mid f(1)=0\}$$
				Wir verwenden wieder das im Aufgabenteil a) erwähnte Teilraumkriterium.\\
				\indent $\bullet$ $\vec{0} \in T$, da $f(1)$ ist auf $0$ abgebildet\\
				\indent $\bullet$ Seinen $f(1)$ und $g(1)$ $\in T$ mit $f(1)=0$ und $g(1)=0$.\\
				Dann gilt: $$f(1)+g(1)= 0+0=0$$\\
				Somit ist $f(1)+g(1) \in T$ und damit $T$ ist abgeschlossen bezüglich der Addition.\\
				\indent $\bullet$ Seinen $f(1) \in T$ und $\alpha \in \mathbb{R}$. Dann gilt:\\
				$$f(1)=0 \hspace{20pt} \text{und} \hspace{20pt} \alpha f(1)= \alpha \cdot 0 = 0 \text{, für alle } \alpha \in \mathbb{R}$$
				Somit ist $T$ abgeschlossen bezüglich skalarer Multiplikation und da slle Bedingungen erfüllt sind, ist $T$ ein Teilraum von $V$.\\
		\noindent \begin{Large}Aufgabe 2:\end{Large}\\[2pt]
			\indent a)\\
				Die Vektoren sind nur dann linear unabhängig, wenn für alle $\lambda_1 \lambda_2 \lambda_3 \in \mathbb{R}$, die folgende Gleichung\\
				$$\lambda_1 \begin{bmatrix}1\\1\\0\end{bmatrix} +
				\lambda_2 \begin{bmatrix}1\\0\\1\end{bmatrix} +
				\lambda_3 \begin{bmatrix}0\\1\\-1\end{bmatrix} = \begin{bmatrix}0\\0\\0\end{bmatrix}$$
				die nür triviale Lösung: $\lambda_1=\lambda_2=\lambda_3=0$ hat.\\
				$$\lambda_1 \begin{bmatrix}1\\1\\0\end{bmatrix} +
				\lambda_2 \begin{bmatrix}1\\0\\1\end{bmatrix} +
				\lambda_3 \begin{bmatrix}0\\1\\-1\end{bmatrix} = \begin{bmatrix}0\\0\\0\end{bmatrix}$$
				$$<=> \begin{bmatrix}\lambda_1\\\lambda_1\\0\end{bmatrix} +
				\begin{bmatrix}\lambda_2\\0\\\lambda_2\end{bmatrix} +
				\begin{bmatrix}0\\\lambda_3 \\-\lambda_3 \end{bmatrix} = \begin{bmatrix}0\\0\\0\end{bmatrix}$$
				$$<=>\begin{bmatrix}\lambda_1+\lambda_2\\\lambda_1+\lambda_3\\\lambda_2-\lambda_3\end{bmatrix}=\begin{bmatrix}0\\0\\0\end{bmatrix}$$
				$$=>\lambda_1+\lambda_2=0=>\lambda_1=-\lambda_2$$
				$$\lambda_1+\lambda_3=0=>\lambda_1=-\lambda_3$$
				$$\lambda_2-\lambda_3=0=>\lambda_2=\lambda_3$$
				Da wir zum Beispiel keine eindeutige Auskunft für $\lambda_1$ haben hat das obige LGS nicht nur die Einzige triviale Lösung $\lambda_1=\lambda_2=\lambda_3=0$.\\
				Deswegen sind unsere Vektoren linear abhängig.\\\
			\indent b)\\
					Wir überprufen die Definition\\
					Sei $\alpha_1$, $\alpha_2$, $\alpha_3$, $\alpha_4$ $\in \mathbb{R}$\\
					$$\alpha_1 p_1(z)+\alpha_2 p_2(z)+\alpha_3 p_3(z)+\alpha_4 p_4(z)=0$$
					$$=>\alpha_1+\alpha_2z+2\alpha_3z^3-\alpha_3+4\alpha_4z^3-3\alpha_4z=0$$
					$$4\alpha_4z^3+2\alpha_3z^2+(\alpha_2-3\alpha_4)z+(\alpha_1-\alpha_3)=0$$
					Wir vergleichen die Koeffizienten:
					$$4\alpha_4=0 \hspace{10pt} => \hspace{10pt} \alpha_4=0$$
					$$2\alpha_3=0 \hspace{10pt} => \hspace{10pt} \alpha_3=0$$
					$$\alpha_2-3\alpha_4=0 \hspace{10pt} => \hspace{10pt} \alpha_2=0$$
					$$\alpha_1-\alpha_3=0 \hspace{10pt} => \hspace{10pt} \alpha_1=0$$
					Somit ist $\alpha_1=\alpha_2=\alpha_3=\alpha_4=0$ die einzige Lösung.\\
					Damit sind $p_1(z)$, $p_2(z)$, $p_3(z)$, $p_4(z)$ linear unabhängig.\\
				\indent c)\\
					Seien $\lambda_1$, $\lambda_2$ $\in \mathbb{R}$ sodass:\\
					$$0=\lambda_1 \cdot f_1(x)+\lambda_2 \cdot f_2(x)$$
					$$<=>0=\lambda_1 \cdot cos(x)+\lambda_2 \cdot cos(2x)$$
					Für spezielle $x$ können wir etwas über $\lambda_1$, $\lambda_2$ herausfinden.\\
					\underline{Für $x=\frac{\pi}{2}$}:\\
					$$0=\lambda_1 \cdot cos \left(\frac{\pi}{2}\right)+\lambda_2 \cdot cos\left(2\frac{\pi}{2}\right)$$
					$$<=>0=\lambda_1 \cdot 0 - \lambda_2$$
					$$<=>\lambda_2=0$$
					\underline{Für $x=\frac{\pi}{4}$}:\\
					$$0=\lambda_1 \cdot cos \left(\frac{\pi}{4}\right)+\lambda_2 \cdot cos\left(2\frac{\pi}{4}\right)$$
					$$0=\lambda_1 \cdot \frac{\sqrt{2}}{2}+\lambda_2 \cdot 0$$
					$$\lambda_1=0$$
					Also Gleichungen:\\
					$$\circled{1} \hspace{10pt} \lambda_2=0$$
					$$\circled{2} \hspace{10pt} \lambda_1=0$$
					Also $\lambda_1=\lambda_2=0$ ist die einzige Lösung des LGS und somit sind $cos(x)$ und $cos(2x)$ linear unabhängig.\\
			\noindent \begin{Large}Aufgabe 3:\end{Large}\\[2pt]
				\indent a)\\
					$$\brkbinom{x}{y}=\alpha \cdot \brkbinom{1}{0}$$
					$$<=>\brkbinom{x}{y}=\brkbinom{\alpha \cdot 1}{\alpha \cdot 0}$$
					$$=>x=\alpha \cdot 1 => x=\alpha$$
					$$=>y=\alpha \cdot 0 => y= x \cdot 0 => y \neq 0$$
					Deswegen $\left\{ \brkbinom{1}{0} \right\}$ ist nicht ein Erzeugendensystem von $\mathbb{R}^2$.\\
				\indent b)\\
					$$\brkbinom{x}{y}=\lambda_1 \brkbinom{2}{1}+\lambda_2 \brkbinom{0}{0}+\lambda_3 \brkbinom{1}{1}+\lambda_4 \brkbinom{0}{1}$$
					$$x=2\lambda_1+\lambda_3 => x=2x-x$$
					$$<=>\lambda_1=x \hspace{40pt} \lambda_3=-x$$
					$$y=\lambda_1+\lambda_3+\lambda_4$$
					$$=>y=x-x+\lambda_4 => \lambda_4=y$$
					$$=x \brkbinom{2}{1}+0 \brkbinom{0}{0}-x \brkbinom{1}{1}+y \brkbinom{0}{1}$$
				\indent c)\\
					$$\brkbinom{x}{y}=x\brkbinom{1}{1}+(y-x)\brkbinom{0}{1}$$
					$$\brkbinom{1}{0}=1 \cdot \brkbinom{1}{1}+(0-1) \cdot \brkbinom{0}{1}=\brkbinom{1}{1}-\brkbinom{0}{1}=\brkbinom{1}{0}$$
			\noindent \begin{Large}Aufgabe 4:\end{Large}\\[2pt]
				\indent a)\\
					Die Addition zweier Matrizen ist nur möglich wenn die Anzahl der Zeilen und Spalten der beiden Matrizen übereinstimmen.\\
					Das heißt also A+C ist m
\end{document}