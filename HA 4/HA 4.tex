\documentclass[11pt]{article}

\usepackage{amsfonts}
\usepackage{graphicx}
\usepackage{amssymb}
\usepackage{tikz}
\usepackage{amsmath}

\newcommand\myeqi{\mathrel{\overset{\makebox[0pt]{\mbox{\normalfont\tiny\sffamily (I)}}}{=}}}
\newcommand\myeqii{\mathrel{\overset{\makebox[0pt]{\mbox{\normalfont\tiny\sffamily (II)}}}{=}}}
\newcommand*\circled[1]{\tikz[baseline=(char.base)]{
            \node[shape=circle,draw,inner sep=2pt] (char) {#1};}}
\DeclareRobustCommand{\brkbinom}{\genfrac[]{0pt}{}}

\begin{document}
	\noindent \begin{huge}Haitham Abdel Razaq Moh'd Almatani  407920\\
	
	\noindent Themistoklis Dimaridis 355835\\
	
	
	\noindent Kirill Beskorovainyi 	451420\end{huge}\\\vspace{0.05in}
	
		\noindent \begin{Large}Aufgabe 1:\end{Large}\\[2pt]
			\indent a)\\
				$$T_1= \left\{\brkbinom{a}{b} \in \mathbb{R}^2 \mid 2x_1=x_2 \right\} \hspace{30pt} T_2= \left\{\brkbinom{a}{b} \in \mathbb{R}^2 \mid x_1^3+x_2^3=1 \right\}$$
				Wir verwenden das Teilraum kriterium. $T_1$, $T_2$ sind Teilräume des $\mathbb{R}^2$, falls diese den Nullvektor enthalten und abgeschlossen bezüglich Addition und skalarer Multiplikation sind.\\
				\underline{Über $T_1$ gilt:}\\
				\indent $\bullet$ $\vec{0}=\brkbinom{0}{0} \in T_1$, da $2 \cdot 0 = 0$\\
				\indent $\bullet$ Für $\brkbinom{x_1}{x_2}$, $\brkbinom{y_1}{y_2} \in T_1$ ist $x_2=2x_1$ und $y_2=2y_1$\\ 
				Dann gilt: $\brkbinom{x_1}{x_2}+\brkbinom{y_1}{y_2}=\brkbinom{x_1+y_1}{x_2+y_2}$ mit\\
				$$x_2+y_2=2x_1+2y_1=2(x_1+y_1)$$
				Also ist $\brkbinom{x_1}{x_2}+\brkbinom{y_1}{y_2} \in T_1$ und damit ist $T_1$ abgeschlossen bezüglich der Addition.\\
				\indent $\bullet$ Sei $\brkbinom{x_1}{x_2} \in T_1$, $\alpha \in \mathbb{R}$. Dann gilt per Definition:\\
				$$x_2=2x_1$$
				Also ist für $\alpha \brkbinom{\alpha x_1}{\alpha x_2}$, dass $\alpha x_2=\alpha (2x_1)=2(\alpha x_1)$\\
				Damit ist $\alpha \brkbinom{x_1}{x_2} \in T_1$ und $T_1$ ist auch abgeschlossen bezüglich skalarer Multiplikation.\\
				Damit ist $T_1$ ein Teilraum des $\mathbb{R}^2$.\\
				
				$T_1\neq\mathbb{R}^2$, da z.B. $\brkbinom{1}{0} \in \mathbb{R}^2$ und $\notin T_1$\\
				Somit ist $dim(T_1)<dim(\mathbb{R}^2)$\\
				$T_1\neq\{\vec{0}\}$, weil z.B. $\brkbinom{3}{6} \in T_1$
				Somit ist $dim(T_1)>0$\\
				Das heißt:\\
				$$0<dim(T_1)<2$$
				$$dim(T_1)=1$$
				\underline{Über $T_2$ gilt:}\\
				\indent $\bullet$ $\vec{0}=\brkbinom{0}{0} \notin T_1$, da $0^3+0^3=0 \neq 1$\\
				$=>$ Somit ist der Nullvektor nicht in $T_2$ enthalten und damit ist $T_2$ kein Teilraum des $\mathbb{R}^2$.\\
			\indent b)\\
				$$V=\{f: \mathbb{R} \longrightarrow \mathbb{R}\}$$
				$$T=\{f: \mathbb{R} \longrightarrow \mathbb{R} \mid f(1)=0\}$$
				Wir verwenden wieder das im Aufgabenteil a) erwähnte Teilraumkriterium.\\
				\indent $\bullet$ $\vec{0} \in T$, da $f(1)$ ist auf $0$ abgebildet\\
				\indent $\bullet$ Seinen $f(1)$ und $g(1)$ $\in T$ mit $f(1)=0$ und $g(1)=0$.\\
				Dann gilt: $$f(1)+g(1)= 0+0=0$$\\
				Somit ist $f(1)+g(1) \in T$ und damit $T$ ist abgeschlossen bezüglich der Addition.\\
				\indent $\bullet$ Seinen $f(1) \in T$ und $\alpha \in \mathbb{R}$. Dann gilt:\\
				$$f(1)=0 \hspace{20pt} \text{und} \hspace{20pt} \alpha f(1)= \alpha \cdot 0 = 0 \text{, für alle } \alpha \in \mathbb{R}$$
				Somit ist $T$ abgeschlossen bezüglich skalarer Multiplikation und da slle Bedingungen erfüllt sind, ist $T$ ein Teilraum von $V$.\\
		\noindent \begin{Large}Aufgabe 2:\end{Large}\\[2pt]
			\indent a)\\
				Die Vektoren sind nur dann linear unabhängig, wenn für alle $\lambda_1 \lambda_2 \lambda_3 \in \mathbb{R}$, die folgende Gleichung\\
				$$\lambda_1 \begin{bmatrix}1\\1\\0\end{bmatrix} +
				\lambda_2 \begin{bmatrix}1\\0\\1\end{bmatrix} +
				\lambda_3 \begin{bmatrix}0\\1\\-1\end{bmatrix} = \begin{bmatrix}0\\0\\0\end{bmatrix}$$
				die nür triviale Lösung: $\lambda_1=\lambda_2=\lambda_3=0$ hat.\\
				$$\lambda_1 \begin{bmatrix}1\\1\\0\end{bmatrix} +
				\lambda_2 \begin{bmatrix}1\\0\\1\end{bmatrix} +
				\lambda_3 \begin{bmatrix}0\\1\\-1\end{bmatrix} = \begin{bmatrix}0\\0\\0\end{bmatrix}$$
				$$<=> \begin{bmatrix}\lambda_1\\\lambda_1\\0\end{bmatrix} +
				\begin{bmatrix}\lambda_2\\0\\\lambda_2\end{bmatrix} +
				\begin{bmatrix}0\\\lambda_3 \\-\lambda_3 \end{bmatrix} = \begin{bmatrix}0\\0\\0\end{bmatrix}$$
				$$<=>\begin{bmatrix}\lambda_1+\lambda_2\\\lambda_1+\lambda_3\\\lambda_2-\lambda_3\end{bmatrix}=\begin{bmatrix}0\\0\\0\end{bmatrix}$$
				$$=>\lambda_1+\lambda_2=0=>\lambda_1=-\lambda_2$$
				$$\lambda_1+\lambda_3=0=>\lambda_1=-\lambda_3$$
				$$\lambda_2-\lambda_3=0=>\lambda_2=\lambda_3$$
				Da wir zum Beispiel keine eindeutige Auskunft für $\lambda_1$ haben hat das obige LGS nicht nur die Einzige triviale Lösung $\lambda_1=\lambda_2=\lambda_3=0$.\\
				Deswegen sind unsere Vektoren linear abhängig.\\\
			\indent b)\\
					Wir überprufen die Definition\\
					Sei $\alpha_1$, $\alpha_2$, $\alpha_3$, $\alpha_4$ $\in \mathbb{R}$\\
					$$\alpha_1 p_1(z)+\alpha_2 p_2(z)+\alpha_3 p_3(z)+\alpha_4 p_4(z)=0$$
					$$=>\alpha_1+\alpha_2z+2\alpha_3z^3-\alpha_3+4\alpha_4z^3-3\alpha_4z=0$$
					$$4\alpha_4z^3+2\alpha_3z^2+(\alpha_2-3\alpha_4)z+(\alpha_1-\alpha_3)=0$$
					Durch Koeffizientenvergleich erhalten wir:\\
					$$\text{(I) }4\alpha_4=0 \hspace{10pt} => \hspace{10pt} \alpha_4=0$$
					$$\text{(II) }2\alpha_3=0 \hspace{10pt} => \hspace{10pt} \alpha_3=0$$
					$$\text{(III) }\alpha_2-3\alpha_4=0 \hspace{10pt} => \hspace{10pt} 3\alpha_4 \myeqi 3\cdot 0=0$$
					$$\text{(IV) }\alpha_1-\alpha_3=0 \hspace{10pt} => \hspace{10pt} \alpha_1=\alpha_3 \myeqii 0$$
					Somit ist $\alpha_1=\alpha_2=\alpha_3=\alpha_4=0$ die einzige Lösung.\\
					Damit sind $p_1(z)$, $p_2(z)$, $p_3(z)$, $p_4(z)$ linear unabhängig.\\
				\indent c)\\
					Seien $\lambda_1$, $\lambda_2$ $\in \mathbb{R}$ sodass:\\
					$$0=\lambda_1 \cdot f_1(x)+\lambda_2 \cdot f_2(x)$$
					$$<=>0=\lambda_1 \cdot cos(x)+\lambda_2 \cdot cos(2x)$$
					Für spezielle $x$ können wir etwas über $\lambda_1$, $\lambda_2$ herausfinden.\\
					\underline{Für $x=\frac{\pi}{2}$}:\\
					$$0=\lambda_1 \cdot cos \left(\frac{\pi}{2}\right)+\lambda_2 \cdot cos\left(2\frac{\pi}{2}\right)$$
					$$<=>0=\lambda_1 \cdot 0 - \lambda_2$$
					$$<=>\lambda_2=0$$
					\underline{Für $x=\frac{\pi}{4}$}:\\
					$$0=\lambda_1 \cdot cos \left(\frac{\pi}{4}\right)+\lambda_2 \cdot cos\left(2\frac{\pi}{4}\right)$$
					$$0=\lambda_1 \cdot \frac{\sqrt{2}}{2}+\lambda_2 \cdot 0$$
					$$\lambda_1=0$$
					Also Gleichungen:\\
					$$\circled{1} \hspace{10pt} \lambda_2=0$$
					$$\circled{2} \hspace{10pt} \lambda_1=0$$
					Also $\lambda_1=\lambda_2=0$ ist die einzige Lösung des LGS und somit sind $cos(x)$ und $cos(2x)$ linear unabhängig.\\
			\noindent \begin{Large}Aufgabe 3:\end{Large}\\[2pt]
				\indent a)\\
					$$\brkbinom{x}{y}=\alpha \cdot \brkbinom{1}{0}$$
					$$<=>\brkbinom{x}{y}=\brkbinom{\alpha \cdot 1}{\alpha \cdot 0}$$
					$$=>x=\alpha \cdot 1 => x=\alpha$$
					$$=>y=\alpha \cdot 0 => y= x \cdot 0 => y \neq 0$$
					Deswegen $\left\{ \brkbinom{1}{0} \right\}$ ist nicht ein Erzeugendensystem von $\mathbb{R}^2$.\\
				\indent b)\\
					$$\brkbinom{x}{y}=\lambda_1 \brkbinom{2}{1}+\lambda_2 \brkbinom{0}{0}+\lambda_3 \brkbinom{1}{1}+\lambda_4 \brkbinom{0}{1}$$
					$$x=2\lambda_1+\lambda_3 => x=2x-x$$
					$$<=>\lambda_1=x \hspace{40pt} \lambda_3=-x$$
					$$y=\lambda_1+\lambda_3+\lambda_4$$
					$$=>y=x-x+\lambda_4 => \lambda_4=y$$
					$$=>x \brkbinom{2}{1}+0 \brkbinom{0}{0}-x \brkbinom{1}{1}+y \brkbinom{0}{1}$$
				\indent c)\\
					$$\brkbinom{x}{y}=x\brkbinom{1}{1}+(y-x)\brkbinom{0}{1}$$
					$$\brkbinom{1}{0}=1 \cdot \brkbinom{1}{1}+(0-1) \cdot \brkbinom{0}{1}=\brkbinom{1}{1}-\brkbinom{0}{1}=\brkbinom{1}{0}$$
			\noindent \begin{Large}Aufgabe 4:\end{Large}\\[2pt]
				\indent a)\\
					Die Addition zweier Matrizen ist nur möglich wenn die Anzahl der Zeilen und Spalten der beiden Matrizen übereinstimmen.\\
					Das heißt also $A+C$ und $B+D$ ist möglich.\\
					$$A+C=\begin{bmatrix}1&4\\-1&1\end{bmatrix}+\begin{bmatrix}-2&38\\3&0\end{bmatrix}=\begin{bmatrix}(1-2)&(4+38)\\(-1+3)&(1+0)\end{bmatrix}=\begin{bmatrix}-1&42\\2&1\end{bmatrix}$$
					$$B+D=\begin{bmatrix}-1&2\end{bmatrix}+\begin{bmatrix}1-3i&-2+6i\end{bmatrix}=$$
					$$\begin{bmatrix}(-1+1-3i)&(2-2+6i)\end{bmatrix}=\begin{bmatrix}-3i&6i\end{bmatrix}$$
				\indent b)\\
					Das Produkt zweier Matrizen ist nur dann definiert, wenn die Anzahl der Spalten der ersten Matrix gleich der Anzahl der Zeilen der zweiten Matrix ist.\\
					Folgende Paare von Matrizen lassen sich also multiplizieren: $A\cdot B$, $A\cdot D$, $B\cdot D$, $C\cdot B$, $C\cdot D$, $D\cdot C$.\\
					$$A\cdot B = \begin{bmatrix}2&2i\end{bmatrix} \cdot \begin{bmatrix}1-i&1\\0&3\end{bmatrix}=\begin{bmatrix}2(1-i)+2i\cdot 0& 2\cdot 1+3\cdot 2i\end{bmatrix}=\begin{bmatrix}2-2i&2+6i\end{bmatrix}$$
					$$A\cdot D=\begin{bmatrix}2&2i\end{bmatrix}\cdot\begin{bmatrix}1-i&2&2-3i\\-1-3i&0&-1\end{bmatrix}$$
					$$=\begin{bmatrix}2(1-i)+2i(-1-3i)&2\cdot 2+2i\cdot 0&2(2-3i)+2i(-1)\end{bmatrix}$$
					$$=\begin{bmatrix}2-2i-2i+6&4&4-6i-2i\end{bmatrix}=\begin{bmatrix}8-4i&4&4-8i\end{bmatrix}$$
					$$B \cdot D=\begin{bmatrix}1-i&1\\0&3\end{bmatrix}\cdot \begin{bmatrix}1-i&2&2-3i\\-1-3i&0&-1\end{bmatrix}$$
					$$=\begin{bmatrix}(1-i)^2-1-3i&2(1-i)&(1-i)(2-3i)-1\\3(-1-3i)&0&-3\end{bmatrix}$$
					$$=\begin{bmatrix}1-2i+i^2-1-3i&2-2i&2-3i-2i+3i^2-1\\-3-9i&0&-3\end{bmatrix}$$
					$$=\begin{bmatrix}-1-5i&2-2i&-2-5i\\-3-9i&0&-3\end{bmatrix}$$
					$$C\cdot B=\begin{bmatrix}i&-1+3i\\2&0\\2+3i&11\end{bmatrix}\cdot \begin{bmatrix}1-i&1\\0&3\end{bmatrix}$$
					$$=\begin{bmatrix}i(1-i)&i+3(-1+3i)\\2(1-i)&2\\(2+3i)(1-i)&2+3i+3 \cdot 11 \end{bmatrix}$$
					$$=\begin{bmatrix}1+i&i-3+9i\\2-2i&2\\2-2i+3i-3i^2&35+3i\end{bmatrix}$$
					$$=\begin{bmatrix}1+i&10i-3\\2-2i&2\\5+i&35+3i\end{bmatrix}$$
					$$C\cdot D= \begin{bmatrix}i&-1+3i\\2&0\\2+3i&11\end{bmatrix}\cdot \begin{bmatrix}1-i&2&2-3i\\-1-3i&0&-1\end{bmatrix}$$
					$$=\begin{bmatrix}i(1-i)-(1-3i)(-1-3i)&2i&i(2-3i)-1(-1+3i)\\2(1-i)&4&4-6i\\(2+3i)(1-i)+11(-1-3i)&4+6i&(2+3i)(2-3i)-11\end{bmatrix}$$
					$$=\begin{bmatrix}1+i-(-1-3i+3i-9)&2i&2i+3+1-3i\\2-2i&4&4-6i\\2-2i+3i+3-11-33i&4+6i&4-6i+6i+9-11\end{bmatrix}$$
					$$=\begin{bmatrix}11+i&2i&4-i\\2-2i&4&4-6i\\-6-32i&4+6i&2\end{bmatrix}$$
					$$D\cdot C=\begin{bmatrix}1-i&2&2-3i\\-1-3i&0&-1\end{bmatrix}\cdot\begin{bmatrix}i&-1+3i\\2&0\\2+3i&11\end{bmatrix}$$
					$$=\begin{bmatrix}(1-i)i+2\cdot 2+(2-3i)(2+3i)&(1-i)(-1+3i)+11(2-3i)\\i(-1-3i)-1(2+3i)&(-1-3i)(-1+3i)-11\end{bmatrix}$$
					$$=\begin{bmatrix}1+i+4+4+6i-6i+9&-1+3i+i+3+22-33i\\-i+3-2-3i&1-3i+3i+9-11\end{bmatrix}$$
					$$=\begin{bmatrix}18+i&24-29i\\1-4i&-1\end{bmatrix}$$
			\noindent \begin{Large}Aufgabe 5:\end{Large}\\[2pt]
				\indent a)\\
					$$A^2=A\cdot A=\begin{bmatrix}0&-1\\1&0\end{bmatrix}\cdot\begin{bmatrix}0&-1\\1&0\end{bmatrix}=\begin{bmatrix}-1&0\\0&-1\end{bmatrix}$$
					$$A^3=A^2\cdot A=\begin{bmatrix}-1&0\\0&-1\end{bmatrix}\cdot\begin{bmatrix}0&-1\\1&0\end{bmatrix}=\begin{bmatrix}0&1\\-1&0\end{bmatrix}$$
					$$A^4=A^3\cdot A=\begin{bmatrix}0&1\\-1&0\end{bmatrix}\cdot\begin{bmatrix}0&-1\\1&0\end{bmatrix}=\begin{bmatrix}1&0\\0&1\end{bmatrix}$$
					$$A \text{ ist invertierbar, da } A\cdot A^{-1}=I \text{, wo } A^{-1}=A^3$$
				\indent b)\\
					$$A^5=A \hspace{20pt} A^6=A^2 \hspace{20pt} A^7=A^3 \hspace{20pt} A^8=A^4 \hspace{20pt} A^9=A^5=A \hspace{20pt} A^{10}=A^6=A^2$$
					$$A^4=I \hspace{20pt} A^3=A^{-1}$$
					$$\text{span}\{A,A^2,A^3,...,A^{10}\}=\text{span}\{A,A^2,A^3,A^4,A,A^2,A^3,A^4,A,A^2\}$$
					$$T=\text{span}\{A,A^2\}=\text{span}\{A,I_2\}$$
					Nach Definition von span bilden $A$,$I_2$ ein EZS von T.\\
					$$\lambda_1\cdot A+\lambda_2\cdot I_2=\begin{bmatrix}0&-\lambda_1\\\lambda_1&0\end{bmatrix}+\begin{bmatrix}\lambda_2&0\\0&\lambda_2\end{bmatrix}=\begin{bmatrix}\lambda_2&-\lambda_1\\\lambda_1&\lambda_2\end{bmatrix}$$
					$$\begin{bmatrix}\lambda_2&-\lambda_1\\\lambda_1&\lambda_2\end{bmatrix}=\begin{bmatrix}0&0\\0&0\end{bmatrix}$$
					$$=>\lambda_1=0$$
					$$=>\lambda_2=0$$
					Somit ist $\{A,I_2\}$ ist ein liner unabhängig Erzeugendensystem von T. Das heißt $\{A,I_2\}$ ist ein Basis von T
					Da $A,I_2$ Dimension von $2\times 2$ hat, ist die Dimension von Basis von $A,I_2$ ist auch $2\times2$.\\
\end{document}