\documentclass[11pt]{article}

\usepackage{amsfonts}
\usepackage{graphicx}
\usepackage{amssymb}
\usepackage{tikz}
\usepackage{amsmath}

\begin{document}
	\noindent \begin{huge}Haitham Abdel Razaq Moh'd Almatani  407920
	
	\noindent Themistoklis Dimaridis 355835
	
	
	\noindent Kirill Beskorovainyi 	451420\end{huge}\\\vspace{0.05in}
	
		\noindent \begin{Large}Aufgabe 1:\end{Large}\\[2pt]
			\indent a) \hspace{20pt} i)\\
			$$z^3 = -8i$$
			Wir bringen erstmal die Zahl $-8i=0+-8i$ in Eulerdarstellung, der Form $z=re^{i\phi}$.\\
			$\bullet$  $r=\sqrt{x^2+y^2}=\sqrt{0^2+(-8)^2}=\sqrt{8^2}=8$\\
			$\bullet$  $\phi = arg(z) = arctan\left(\frac{Im(z)}{Re(z)}\right)$\\
			\indent \indent In unserem Fall der Winkel $\phi = -\frac{\pi}{2}$, da $x=0$ und $y<0$.\\
			Also gilt:\\
			$$-8i=re^{i\phi}$$
			$$-8i=8e^{-\frac{\pi}{2}i}$$
			Das heißt also:\\
			$$z^3=-8i$$
			$$<=>z^3=8e^{-\frac{\pi}{2}i}$$
			$$=> r^3 = 8 \hspace{40pt} 3\phi = -\frac{\pi}{2}+2k\pi \mbox{, } k\in \mathbb{Z}$$
			$$r=\sqrt[3]{8}=2 \hspace{40pt} \phi = -\frac{\pi}{6}+\frac{2k\pi}{3} \mbox{, } k\in \mathbb{Z}$$
			Für $k=0,1,2$ bekommen wir unsere $3$ verschiedenen Lösungen wie folgendes:\\
			\indent $k=0$ \hspace{10pt}: \hspace{10pt} $y_0=-\frac{\pi}{6}$ und somit $z_0=2e^{-\frac{\pi}{6}i}$\\
			\indent $k=1$ \hspace{10pt}: \hspace{10pt} $y_1=-\frac{\pi}{6} + \frac{2\pi}{3}=\frac{3\pi}{6}=\frac{\pi}{2}$ und somit $z_1=2e^{\frac{\pi}{2}i}$\\
			\indent $k=2$ \hspace{10pt}: \hspace{10pt} $y_2=-\frac{\pi}{6}+\frac{2\times 2 \pi}{3}=-\frac{\pi}{6}+\frac{4\pi}{3}=\frac{7\pi}{6}$ und somit $z_2=2e^\frac{7\pi}{6}i$\\
			Allgemeine Polardarstellung:\\
			$$z=r(cos(\phi)+isin(\phi))$$
\newpage
			Somit sehen unsere Lösungen wie folgt aus:\\
			\indent $z_0=2(cos\left(-\frac{\pi}{6}\right)+isin\left(-\frac{\pi}{6}\right))$\\
			\indent $z_1=2(cos\left(\frac{\pi}{6}\right)+isin\left(\frac{\pi}{6}\right))$\\
			\indent $z_2=2(cos\left(\frac{7\pi}{6}\right)+isin\left(\frac{7\pi}{6}\right))$\\
			
\end{document}