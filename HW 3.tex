\documentclass[11pt]{article}

\usepackage{amsfonts}
\usepackage{graphicx}
\usepackage{amssymb}
\usepackage{tikz}
\usepackage{amsmath}
\usepackage{polynom}

\newcommand\mybar{\kern1pt\rule[-\dp\strutbox]{.8pt}{\baselineskip}\kern1pt}
\newcommand*\circled[1]{\tikz[baseline=(char.base)]{
            \node[shape=circle,draw,inner sep=2pt] (char) {#1};}}

\begin{document}
	\noindent \begin{huge}Haitham Abdel Razaq Moh'd Almatani  407920
	
	\noindent Themistoklis Dimaridis 355835
	
	
	\noindent Kirill Beskorovainyi 	451420\end{huge}\\\vspace{0.05in}
	
		\noindent \begin{Large}Aufgabe 1:\end{Large}\\[2pt]
			\indent a) \hspace{20pt} i)\\
				$$z^3 = -8i$$
				Wir bringen erstmal die Zahl $-8i=0+(-8i)$ in Eulerdarstellung, der Form $z=re^{i\phi}$.\\
				$\bullet$  $r=\sqrt{x^2+y^2}=\sqrt{0^2+(-8)^2}=\sqrt{8^2}=8$\\
				$\bullet$  $\phi = arg(z) = arctan\left(\frac{Im(z)}{Re(z)}\right)$\\
				\indent \indent In unserem Fall der Winkel $\phi = -\frac{\pi}{2}$, da $x=0$ und $y<0$.\\
				Also gilt:\\
				$$-8i=re^{i\phi}$$
				$$-8i=8e^{-\frac{\pi}{2}i}$$
				Das heißt also:\\
				$$z^3=-8i$$
				$$<=>z^3=8e^{-\frac{\pi}{2}i}$$
				$$=> r^3 = 8 \hspace{40pt} 3\phi = -\frac{\pi}{2}+2k\pi \mbox{, } k\in \mathbb{Z}$$
				$$r=\sqrt[3]{8}=2 \hspace{40pt} \phi = -\frac{\pi}{6}+\frac{2k\pi}{3} \mbox{, } k\in \mathbb{Z}$$
				Für $k=0,1,2$ bekommen wir unsere $3$ verschiedenen Lösungen wie folgendes:\\
				\indent $k=0$ \hspace{10pt}: \hspace{10pt} $\phi_0=-\frac{\pi}{6}$ und somit $z_0=2e^{-\frac{\pi}{6}i}$\\
				\indent $k=1$ \hspace{10pt}: \hspace{10pt} $\phi_1=-\frac{\pi}{6} + \frac{2\pi}{3}=\frac{3\pi}{6}=\frac{\pi}{2}$ und somit $z_1=2e^{\frac{\pi}{2}i}$\\
				\indent $k=2$ \hspace{10pt}: \hspace{10pt} $\phi_2=-\frac{\pi}{6}+\frac{2\times 2 \pi}{3}=-\frac{\pi}{6}+\frac{4\pi}{3}=\frac{7\pi}{6}$ und somit $z_2=2e^\frac{7\pi}{6}i$\\
				Allgemeine Polardarstellung:\\
				$$z=r(cos(\phi)+isin(\phi))$$
\newpage
				\noindent Somit sehen unsere Lösungen wie folgt aus:\\
				\indent $z_0=2(cos\left(-\frac{\pi}{6}\right)+isin\left(-\frac{\pi}{6}\right))$\\[5pt]
				\indent $z_1=2(cos\left(\frac{\pi}{2}\right)+isin\left(\frac{\pi}{2}\right))$\\[5pt]
				\indent $z_2=2(cos\left(\frac{7\pi}{6}\right)+isin\left(\frac{7\pi}{6}\right))$\\[5pt]
			\indent a) \hspace{20pt} ii)\\
				$$z^4+2(\sqrt{12}-2i)z^2+8-4\sqrt{12}i=0$$
				$$z^4+2\times z^2(\sqrt{12}-2i)+(\sqrt{12}-2i)^2=0$$
				Laut der binomischen Formel: $(a+b)^2 = a^2+2ab+b^2$ können wir die Gleichung umformen:\\
				$$(z^2+\sqrt{12}-2i)^2=0$$
				$$<=>z^2+\sqrt{12}-2i=0$$
				$$<=>z^2=-\sqrt{12}+2i$$
				$-\sqrt{12}+2i$ bringen wir zunächst in Eulerdarstellung.\\
				$r=\sqrt{x^2+y^2}=\sqrt{(-12)^2+2^2}=\sqrt{12+4}=\sqrt{16}=4$\\
				$\phi = arg(z) = arctan\left(\frac{Im(z)}{Re(z)}\right) + \pi$, da $x<0$.\\
				\indent Also $arctan \left( \frac{2}{-\sqrt{12}}\right)+\pi = arctan\left(\frac{2}{-2\sqrt{3}}\right)+\pi = arctan \left(-\frac{\sqrt{3}}{3}\right)+\pi = -\frac{\pi}{6}+\pi = \frac{5\pi}{6}$\\
				Also gilt:
				$$-\sqrt{12}+2i=4e^{i\frac{5\pi}{6}}$$
				$z^2$ in Eulerdarstellung: $z^2=(re^{i\phi})^2=r^2e^{i2\phi}$\\
				Also gilt:\\
				$$z^2=-\sqrt{12}+2i$$
				$$<=>r^2e^{i2\phi}=4e^{i\frac{5\pi}{6}}$$
				$$=> r^2 =4 \hspace{20pt} \mbox{und} \hspace{20pt} 2\phi = \frac{5\pi}{6}+2k\pi \mbox{, } k\in\mathbb{Z}$$
				$$r=2 \hspace{45pt} \phi = \frac{5\pi}{12}+k\pi \mbox{, } k\in\mathbb{Z}$$
				Also für $k=0,1$ bekommen wir unsere 2 verschiedene Lösungen wie folgendes:\\
				\indent $k=0$: \hspace{10pt} $\phi _0 = \frac{5\pi}{12}$ und somit $z_0= 2e^{\frac{5\pi}{12}i}$\\
				\indent $k=1$: \hspace{10pt} $\phi _1 = \frac{5\pi}{12} + \pi = \frac{17\pi}{12}$ und somit $z_1=2e^{\frac{17\pi}{12}i}$\\
				Allgemeine Polardarstellung:\\
				$$z=r(cos(\phi)+isin(\phi))$$
				Somit sehen unsere Lösungen wie folgt aus:\\
				\indent $z_0=2(cos(\frac{5\pi}{12}+isin(\frac{5\pi}{12}))$\\
				\indent $z_1=2(cos(\frac{17\pi}{12}+isin(\frac{17\pi}{12}))$\\
			\indent b)\\
				$$z_1=\sqrt{2}e^{-\frac{\pi}{4}i}$$
				$z_1$ in allgemeine Eulerdarstellung:
				$$z_1=re^{i\phi}$$
				Also gilt: $re^{i\phi} = \sqrt{2}e^{-\frac{\pi}{4}i}$\\
				$$=> r=\sqrt{2} \hspace{30pt} \phi=-\frac{\pi}{4}$$
				Also ist $z_1$ in Polardarstellung:
				$$z_1=r(cos(\phi)+isin(\phi))$$
				$$z_1=\sqrt{2}(cos\left(-\frac{\pi}{4}\right)+isin\left(-\frac{\pi}{4}\right))$$
				$$z_1=\sqrt{2}\left(\frac{\sqrt{2}}{2}-i\frac{\sqrt{2}}{2}\right)$$
				$$z_1=1-i$$
				Also $z_1=1-i$ in kartesische Darstellung.\\
			\indent c)\\
				Sei $z=e^{\frac{5\pi}{12}i}$ in Eulerdarstellung.\\
				Allgemein gilt: $z=re^{i\phi}$\\
				Also ist $re^{i\phi}=e^{\frac{5\pi}{12}i}$\\
				$=> r=1$ und $\phi = \frac{5\pi}{12}=\frac{\pi}{4}+\frac{\pi}{6}$\\
				Für die Polardarstellung gilt:\\
				$$z=r(cos(\phi)+isin(\phi))$$
				$$1\times(cos\left(\frac{5\pi}{12}\right)+isin\left(\frac{5\pi}{12}\right))$$
				$$z=\left(cos\left(\frac{\pi}{4}+\frac{\pi}{6}\right)+isin\left(\frac{\pi}{4}+\frac{\pi}{6}\right)\right)$$
				Allgemein gilt für die Additionstheoreme:\\
				\indent $cos(x+y)=cos(x)cos(y)-sin(x)sin(y)$\\
				\indent $sin(x+y)=sin(x)cos(y)+cos(x)sin(y)$\\
				Mithilfe also der Additionstheoreme haben wir:
				$$z=cos\left(\frac{\pi}{4}\right)\times cos\left(\frac{\pi}{6}\right)-sin\left(\frac{\pi}{4}\right)\times sin\left(\frac{\pi}{6}\right)+i\left(sin\left(\frac{\pi}{4}\right)\times cos\left(\frac{\pi}{6}\right)+cos\left(\frac{\pi}{4}\right)\times sin\left(\frac{\pi}{6}\right)\right)$$
				$$z=\left(\frac{\sqrt{2}}{2}\times \frac{\sqrt{3}}{2}\right)-\left(\frac{\sqrt{2}}{2}\times \frac{1}{2}\right)+i\left(\frac{\sqrt{2}}{2}\times \frac{\sqrt{3}}{2}+\frac{\sqrt{2}}{2}\times \frac{1}{2}\right)$$
				$$z=\frac{\sqrt{6}}{4}-\frac{\sqrt{2}}{4}+i\left(\frac{\sqrt{6}}{4}+\frac{\sqrt{2}}{4}\right)$$
				$$z=\frac{\sqrt{6}-\sqrt{2}}{4}+i\left(\frac{\sqrt{6}+\sqrt{2}}{4}\right) \mbox{in kartesische Darstellung}$$
		\noindent \begin{Large}Aufgabe 2:\end{Large}\\[2pt]
				$$p(z)=z^4+z^3+3z^2+4z-4$$
			\indent a)\\
				$$p(2i)=(2i)^4+(2i)^3+3(2i)^2+4(2i)-4$$
				$$=2^4\cdot i^4+2^3\cdot i^3+3\cdot 4i^2+8i-4$$
				$$=16(1) +8(-i)+12(-1)+8i-4$$
				$$=16-12-4+8i-8i$$
				$$=0$$
			\indent b)\\
				$$q(z)=z^2+4$$
				\polyset{vars=z}
				\polylongdiv[style=C]{z^4+z^3+3z^2+4z-4}{z^2+4}\\
			\indent c)\\
				Bestimmung der Nullstellen von $z^2+z-1$ und $x^2+4$:\\
				$$z^2+4 =0 \hspace{120pt} z^2+z-1=0$$
				$$<=>z^2=-4 \hspace{80pt} z_{1,2}=-\frac{p}{2}\pm \sqrt{\left(\frac{p}{2}\right)^2-q}$$
				$$<=>z^2=4i^2 \hspace{90pt} =-\frac{1}{2}\pm \sqrt{\left(\frac{1}{2}\right)^2+1}$$
				$$<=>z=\pm \sqrt{4i^2} \hspace{120pt} =-\frac{1}{2}\pm \sqrt{\frac{5}{4}}$$
				$$<=>z=\pm 2i \hspace{140pt} =-\frac{1}{2} \pm \frac{\sqrt{5}}{2}$$
				Also die komplexe Linearfaktorzerlegung von $p(z)$:\\
				$$p(z)=z^4+z^3+3z^2+4z-4=(z-2i)(z+2i)(z-\left(-\frac{1}{2}+\frac{\sqrt{5}}{2}\right))(z-\left(-\frac{1}{2}-\frac{\sqrt{5}}{2}\right))$$
			\indent d)\\
				Die reelle Zerlegung von $p(z)$:\\
				$$p(z)=z^4+z^3+3z^2+4z-4=(z^2+4)(z-\left(-\frac{1}{2}+\frac{\sqrt{5}}{2}\right))(z-\left(-\frac{1}{2}-\frac{\sqrt{5}}{2}\right))$$
		\noindent \begin{Large}Aufgabe 3:\end{Large}\\[2pt]
			\indent a)\\
				$$f(x)=\frac{3x-1}{x^2-4x+13}$$
				Wir müssen erstmal die Nullstellen des Nenners, $x^2-4x+13$, bestimmen.\\
				$$x^2-4x+13=0$$
				$$\mbox{p-q Formel: } x_{1,2}=-\frac{p}{2}\pm \sqrt{\left(\frac{p}{2}\right)^2-q}$$
				$$=2\pm \sqrt{2^2-13}$$
				$$=2\pm \sqrt{-9}$$
				$$=2 \pm \sqrt{9i^2}$$
				$$=2\pm 3i$$
				Also $f(x)=\frac{3x-1}{x^2-4x+13}=\frac{3x-1}{(x-(2+3i))(x-(2-3i))}$\\
				Also der Ansatz für die komplexe Partialbruchzerlegung von $f(x)$ lautet:
				$$\frac{3x-1}{x^2-4x+13}=\frac{A}{x-(2+3i)}+\frac{B}{x-(2-3i)}$$
			\indent b)\\
				$$f(x)=\frac{2+i}{x-(3+2i)}+\frac{2-i}{x-(3-2i)}$$
				$$=\frac{(2+i)(x-(3-2i))+(2-i)(x-(3+2i))}{(x-(3+2i))(x-(3-2i))}$$
				$$=\frac{(2+i)(x-3+2i)+(2-i)(x-3-2i)}{(x-3-2i)(x-3+2i)}$$
				$$=\frac{2x-6+4i+xi-3i-2+2x-6-4i-xi+3i-2}{x^2-3x+2xi-3x+9-6i-2xi+6i+4}$$
				$$=\frac{4x-16}{x^2-6x+13}$$
		\noindent \begin{Large}Aufgabe 4:\end{Large}\\[2pt]
			\indent a)\\
				$$\frac{11x+18}{x^2+x-6}=\frac{p(x)}{q(x)}$$
				$1=deg(p)<deg(q)=2$, also keine Polynomdivision nötig.\\
				Bestimme die Nullstellen des Nenners mit p-q Formel:\\
				$$x_{1,2}=-\frac{p}{2}\pm \sqrt{\left(\frac{p}{2}\right)^2-q}$$
				$$=-\frac{1}{2}\pm\sqrt{\frac{1}{4}+6}$$
				$$=-\frac{1}{2}\pm\sqrt{\frac{25}{4}}$$
				$$=-\frac{1}{2}\pm\frac{5}{2}$$
				Also:\\
				$$x_1=-\frac{1}{2}+\frac{5}{2}=2$$
				$$x_2=-\frac{1}{2}-\frac{5}{2}=-3$$
				Also gilt:\\
				$$\frac{11x+18}{x^2+x-6}=\frac{11x+18}{(x-2)(x+3)}$$
				$$\frac{11x+18}{(x-2)(x+3)}=\frac{A}{x-2}+\frac{B}{x+3} \hspace{20pt} \mybar \times (x-2)(x+3)$$
				$$<=>11x+18=\frac{A(x-2)(x+3)}{(x-2)}+\frac{B(x-2)(x+3)}{(x+3)}$$
				$$<=>11x+18=Ax+3A+Bx-2B$$
				$$<=>11x+18=(A+B)x+3A-2B$$
				Durch Koeffizientenvergleich erhalten wir:\\
				$$x^1 \mbox{ : } \hspace{20pt} 11=A+B=>A=11-B \hspace{10pt} \mbox{\circled{1}}$$
				$$x^0 \mbox{ : } \hspace{20pt} 18=3A-2B \hspace{10pt} \mbox{\circled{2}}$$
				\indent \circled{1} in \circled{2}: \hspace{50pt} $18=3(11-B)-2B$
				$$<=>18=33-3B-2B$$
				$$<=>33-18=5B$$
				$$<=>B=\frac{33-18}{5}$$
				$$<=>B=\frac{15}{5}=>B=3$$
				\indent $B=3$ in \circled{1}:\\
				$$A=11-3$$
				$$A=8$$
				In unserem Fall, da unsere Nullstellen reelle Zahlen sind, stimmen die reele und komplexe Partialbruchzerlegung überein und somit ist:\\
				$$\frac{11x+18}{x^2+x-6}=\frac{8}{x-2}+\frac{3}{x+3}$$
			\indent b)\\
				$$\frac{x^3-4x^2-2x+17}{x^2-6x+9}$$
				Sei $p(x)=x^3-4x^2-2x+17$\\
				und $q(x)=x^2-6x+9$\\
				Da $3=deg(p)>deg(q)=2$ ist hier eine Polynomdivision notwendig.\\
				\polyset{vars=x}
				\polylongdiv[style=C]{x^3-4x^2-2x+17}{x^2-6x+9}\\
				Also:\\
				$$\frac{x^3-4x^2-2x+17}{x^2-6x+9}=\frac{(x+2)(x^2-6x+9)+x-1}{x^2-6x+9}=x+2+\frac{x-1}{x^2-6x+9}$$
				Das heißt also, wir müssen nun die Partialbruchzerlegung für $\frac{x-1}{x^2-6x+9}$ bestimmen.
				$\frac{x-1}{x^2-6x+9}=\frac{x-1}{(x-3)^2}$ \hspace{10pt} Das heißt also, wir haben die Polstelle 3 mit Vielfachheit 2.\\
				Der Ansatz für die Partialbruchzerlegung:\\
				$$\frac{x-1}{(x-3)^2}=\frac{A}{x-3}+\frac{B}{(x-3)^2} \hspace{10pt} \mybar \times (x-3)^2$$
				$$=>x-1=A(x-3)+B$$
				$$x-1=Ax+B-3A$$
				Durch Koeffizientenvergleich:\\
				\indent $x^1$: $1=A$ \circled{1}\\
				\indent $x^0$: $-1=B-3A$ \circled{2}\\
				\circled{1} in \circled{2}: $-1=B-3(1) => B=2$\\
				In unserem Fall stimmen wieder reelle und komplexe Partialbruchzerlegung überein mit:\\
				$$\frac{x^3-4x^2-2x+17}{x^2-6x+9}=x+2+\frac{1}{x-3}+\frac{2}{(x-3)^2}$$
			\indent c)\\
				$$\frac{2x^2-2x-1}{x^3+2x}=\frac{p(x)}{q(x)}$$
				Keine Polynomdivision nötig, da $2=deg(p)<deg(q)=3$.\\
				$$\frac{2x^2-2x-1}{x^3+2x}=\frac{2x^2-2x-1}{x(x^2+2)}$$
				Bestimmung der Nullstellen des Nenners $x^2+2$:\\
				$$x^2+2=0$$
				$$x^2=-2$$
				$$x^2=2i^2$$
				$$x=\pm \sqrt{2}i$$
				Also gilt:\\
				$$\frac{2x^2-2x-1}{x(x^2+2)}=\frac{2x^2-2x-1}{x(x-\sqrt{2}i)(x+\sqrt{2}i)}$$
				Der Anzatz für die komplexe Partialbruchzerlegung ist:\\
				$$\frac{2x^2-2x-1}{x(x-\sqrt{2}i)(x+\sqrt{2}i)}=\frac{A}{x}+\frac{B}{x-\sqrt{2}i}+\frac{C}{x+\sqrt{2}i}$$
				Bestimmung der Koeffizienten $A,B,C$ mithilfe der Zuhaltemethode:\\
				$$A=\frac{2 \cdot 0^2 -2\cdot 0-1}{(0-\sqrt{2}i)(0+\sqrt{2}i)}=\frac{-1}{-\sqrt{2}^2i^2}=-\frac{1}{2}$$
				$$B=\frac{2 \cdot (\sqrt{2}i)^2-2\sqrt{2}i-1}{\sqrt{2}i \cdot 2 \cdot \sqrt{2}i}=\frac{-4-2\sqrt{2}i-1}{2 \cdot 2 \cdot i^2}=\frac{-5-2\sqrt{2}i}{-4}$$
				$$=>B=\frac{5}{4}+\frac{1}{2}\sqrt{2}i$$
				$$C=\frac{2(-\sqrt{2}i)^2-2(-\sqrt{2}i)-1}{(-\sqrt{2}i)(-2\sqrt{2}i)}=\frac{2(-2)+2\sqrt{2}i-1}{2\sqrt{2}^2i^2}=\frac{-5+2\sqrt{2}i}{-2 \cdot 2}=\frac{-(5-2\sqrt{2}i)}{-4}$$
				$$=>C=\frac{5}{4}-\frac{1}{2}\sqrt{2}i$$
				Damit ist die komplexe Partialbruchzerlegung:\\
				$$\frac{2x^2-2x-1}{x^3+2x}=\frac{-\frac{1}{2}}{x}+\frac{\frac{5}{4}+\frac{1}{2} \cdot \sqrt{2}i}{x-\sqrt{2}i}+\frac{\frac{5}{4}-\frac{1}{2}\sqrt{2}i}{x+2\sqrt{2}i}$$
				Der Ansatz für die reelle Partialbruchzerlegung lautet:\\
				$$\frac{2x^2-2x-1}{x^3+2x}=\frac{A}{x}+\frac{Bx+C}{x^2+2} \hspace{10pt} \mybar \times x(x^2+2)$$
				$$=>2x^2-2x-1=A(x^2+2)+(Bx+C)\times x$$
				$$<=>2x^2-2x-1=Ax^2+2A+Bx^2+Cx$$
				$$<=>2x^2-2x-1=(A+B)x^2+Cx+2A$$
				Koeffizientenvergleich:\\
				\indent $x^2$: $2=A+B$ \hspace{70pt} \circled{1}\\
				\indent $x^1$: $-2=C$ \hspace{80pt} \circled{2}\\
				\indent $x^0$: $-1=2A=>A=-\frac{1}{2}$ \hspace{20pt}\circled{3}\\
				\circled{3} in \circled{1} einsetzen:\\
				$$2=A+B$$
				$$B=2-A$$
				$$B=2-(-\frac{1}{2})=2+\frac{1}{2}=\frac{5}{2}$$
				Somit ist die reelle Partialbruchzerlegung:
				$$\frac{2x^2-2x-1}{x^3+2x}=\frac{-\frac{1}{2}}{x}+\frac{\frac{5}{2}x-2}{x^2+2}$$
\end{document}